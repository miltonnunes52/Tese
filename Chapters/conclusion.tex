\chapter{Conclusões e Trabalho Realizado}

Um Ambiente Inteligente consiste, tal como já foi abordado nesta dissertação, na utilização da informação ambiente de um ser humano para realizar tarefas ou disponibilizar serviços com o objetivo de promover a sua comodidade e bem estar. Foi com base nesta definição que este trabalho se centrou na utilização da informação ambiente do próprio ser humano para lhe fornecer um serviço. Esse serviço trata-se um serviço de monitorização das suas ações e comportamentos que apresenta aos seus utilizadores a informação proveniente dos períodos em que estes são monitorizados. Este tipo de serviços tem normalmente condicionantes relacionados com a utilização dos sensores, das suas tecnologias e do tipo de sensor que se trata. Neste trabalho um dos objetivos era ultrapassar essas limitações e construir uma solução que oferecesse uma solução o mais ampla e vasta possível neste aspeto.

A solução consistiu então no desenvolvimento de uma arquitetura composta por sensores colocados de forma distribuída que permitisse oferecer um serviço sem estas limitações. Para tal, a arquitetura desenvolvida foi baseada no conceito de interoperabilidade. A interoperabilidade consiste na capacidade de diversos componentes comunicarem de forma transparente garantindo assim a cooperação entre eles, independentemente das suas características e tecnologias. Esta é uma das chaves desta arquitetura e uma das grandes mais valias deste trabalho. Deste modo, os métodos de comunicação implementados permitem que sensores de vários tipos consigam integrar facilmente a arquitetura e contribuir com a recolha de informação sobre os seus utilizadores. Esta solução permite desta forma a recolha de informação usando diversos dispositivos, independentemente de características como o local de monitorização ou o contexto em questão.

A arquitetura desenvolvida foi assim construída de modo a oferecer uma série de funcionalidades e apresentar um conjunto de características que garantam o seu correto funcionamento. Foi ainda efetuada uma experiência que permitiu validar as funcionalidades da arquitetura e o seu correto funcionamento, bem como analisar os resultados obtidos na demonstração. Para além disso, a interface desenvolvida, com o intuito de visualizar os resultados provenientes da monitorização e do processo de tratamento da informação, permite também confirmar a funcionalidade da arquitetura. Através da interface cada utilizador tem assim acesso os seus registos de monitorização e pode ainda verificar que o resultado final da arquitetura corresponde aos objetivos delineados.


\section{Trabalho Realizado}

Do trabalho desenvolvido durante esta dissertação resultaram várias contribuições importantes. Podem então essas contribuições ser resumidas da seguinte forma:

\begin{itemize}
	\item Análise e estudo de Ambientes Inteligentes e da sua contribuição para a realização de tarefas humanas ou na disponibilização de serviços. Foi também estudado o conceito de Monitorização, aliada a utilização de sensores enquadrados numa Arquitetura Orientada a Serviços.
	\item Desenvolvimento de uma arquitetura para a recolha de informação proveniente de sensores colocados de forma distribuída. Arquitetura que inclui métodos de comunicação que lhe permitem recolher a informação, processá-la através da integração de Serviços de Métricas e armazenar a informação resultante.
	\item Deste trabalho resultou a seguinte publicação: Gomes M, Carneiro D., Pimenta A, Nunes M., Novais P., Neves J., Improving Modularity, Interoperability and Extensibility in Ambient Intelligence , Ambient Intelligence- Software and Applications – 5th International Symposium on Ambient Intelligence (ISAmI 2014), Carlos Ramos, Paulo Novais, Céline Ehrwein Nihan, Juan M. Corchado Rodríguez (eds), Springer - Series Advances in Intelligent and Soft Computing, Vol 219, ISBN 978-3-319-07596-9, pp 63-70, 2014.
	\item Elaborada uma experiência com várias condições que permitiu validar a funcionalidade, interoperabilidade e a utilidade da arquitetura para os seus utilizadores.
	\item Desenvolvimento de uma interface que permite aos utilizadores da arquitetura aceder aos seus registos de monitorização e informações relativas à sua utilização do sistema. E através da qual os administradores podem consultar todos os dados relacionados com o funcionamento da arquitetura.

\end{itemize}

O trabalho desenvolvido foca sobretudo a área de Ambientes Inteligentes com o intuito de oferecer aos seus utilizadores um sistema de monitorização sob a forma de arquitetura composta por sensores distribuídos, através da qual podem recolher informação e aceder aos resultados alcançados.


\section{Trabalho Futuro}

O trabalho desenvolvido foi bastante satisfatório e abrangente relativamente aos objetivos delineados. Contudo existem sempre pontos que podem ser melhorados, independentemente do trabalho em questão. Neste em particular existem algumas melhorias que se tornariam claras mais valias para esta arquitetura e que aumentariam a sua capacidade.

Assim, os pontos a realizar como trabalho futuro deste trabalho são os seguintes:

\begin{itemize}
	\item Adicionar novos serviços de métricas associados, sobretudo, a novos tipos de dados.
	\item Desenvolver novas funcionalidades de administração na interface, como editar determinados registos e adicionar serviços de métricas através desta.
	\item Adicionar contextos aos períodos de monitorização, situação para a qual a base de dados está preparada, que permitam identificar o contexto em que cada monitorização ocorreu.
	\item Acrescentar parâmetros de segurança sobre a informação e comunicações estabelecidas.
\end{itemize}

Este conjunto de pontos ficam como trabalho futuro a realizar brevemente, de forma a melhorar o trabalho aqui documentado.