\chapter{Ambientes Inteligentes}

\section{O que são?}

Um Ambiente Inteligente (AmI) \cite{ducatel2001scenarios, ducatel2003ambient} consiste na utilização da informação existente no ambiente e que rodeia um ser humano de forma a facilitar a realização de tarefas ou disponibilizando serviços com o intuito promover a sua comodidade e bem estar. Este conceito relativamente recente é visto como uma área ainda em desenvolvimento mas à qual se vaticina um grande impacto no futuro, suscitando elevado interesse em diversas áreas da indústria e saúde, por exemplo \cite{aarts2006into}.

Esta área baseia-se num novo conceito de processamento e utilização de informação ambiente no qual a utilização de vários dispositivos industriais, de investigação científica, elétricos e ligados à computação são uma realidade e apenas conjuntamente é possível a construção de Ambientes Inteligentes. Este suporte proporcionado por aparelhos tecnológicos de diversos tipos é fundamental para tarefas como a identificação, recolha de dados e monitorização da ação humana e do ambiente \cite{ducatel2001scenarios}. Desta forma, os AmI usufruem do desenvolvimento tecnológico corrente para promover a integração de vários dispositivos num ambiente e poder retirar através destes partido da informação ambiente abundante.

Um foco fundamental em Ambientes Inteligentes é a interação com o ser humano \cite{aarts2006into}. O sucesso de um AmI depende muitas vezes da capacidade de interação que um ser humano consegue ter com o sistema, existindo para isso a necessidade de este oferecer um suporte adequado para que este fator não se revele um problema. Assim, as interfaces com o ser humano devem ser um ponto de especial atenção. A facilidade de um utilizador em interagir com o sistema deve ser uma prioridade devendo existir um suporte adequando através de implementação de interfaces que permitam uma utilização o mais intuitiva possível, que proporcionem um ambiente no qual o ser humano se sinta confortável e que idealmente não afete o seu normal comportamento.

Outra características de um AmI é a adaptatividade que este deve oferecer \cite{aarts2006into}. Cada ser humano tem as suas características e preferências e um Ambiente Inteligente deve ser sensível a isso. Este deve ser desenvolvido de forma a adaptar-se às rotinas dos seus utilizadores, tendo em conta as preferências de cada um, com o intuito de auxiliar o ser humano na realização das suas tarefas, através da simplificação e diminuição de esforço aplicado na realização das mesmas.

Fornecer ao ser humano um conjunto de serviços ou métodos que permitam melhorar as suas condições de vida e de bem estar e dar suporte à realização das suas tarefas quotidianas não é um objetivo fácil, significa um grande trabalho na evolução dos Ambientes Inteligentes desde o seu aparecimento até então. São vários os fatores a ter em conta e a integrar para que funcionando de forma coletiva fosse possível desenvolver esta nova abordagem: um ambiente que permita incrementar simplicidade nas tarefas, ações e na interação com o seu utilizador.

\section{Informação Ambiente}
A Informação Ambiente \cite{bentley2007time} consiste em informação que se encontra dispersa no ambiente, podendo esta ser proveniente do ser humano pela sua interação com o ambiente ou pelo normal decorrer das suas atividades, e que não é normalmente visível a olho nu ou inferível. Esta informação foi durante muito tempo ignorada e posteriormente desaproveitada, isto pela dificuldade da sua perceção por parte do ser humano e depois disso pela incapacidade de recolha e registo desta informação.

O facto deste tipo de informação não ser habitualmente fácil de identificar implica a utilização de métodos e ferramentas próprias para a sua identificação, levantamento e por vezes transformação. Isto implica que sejam utilizados dispositivos tecnológicos como os sensores, desenvolvidos especificamente para essa função \cite{streitz2003situated}. A sua utilização é fundamental por exemplo na construção de um Ambiente Inteligente, pois permite o levantamento da informação ambiente existente para posteriormente esta ser tratada e utilizada em prol do ser humano\cite{pimenta2013monitoring}. 

A Informação Ambiente pode ser vista então como a fonte de informação privilegiada do trabalho, sendo esta recolhida com recurso a dispositivos tecnológicos construídos para o efeito. É possível assim concluir que apesar de não ser vísivel aos olhos humanos a Informação Ambiente abunda e a construção de um ambiente controlado em que esta é recolhida, analisada e utilizada em função do utilizador permite facilitar um conjunto alargado de tarefas quotidianas, bem como registar avanços significativos em várias áreas como a saúde, militar e até mesmo a nível de investigação científica. Tudo isto com recurso a informação que sempre existiu mas para a qual ainda não havia nenhuma metodologia que permitisse uma utilização adequada.

\section{Ambiente Inteligente como Prestação de Serviços}
Um dos propósitos de um Ambiente Inteligente pode ser fornecer um conjunto de serviços aos seus utilizadores, ou seja, é possível desenvolver um Ambiente Inteligente em que o resultado oferecido aos seus utilizadores sejam sob a forma de serviços que facilitem, por exemplo, a execução de algumas tarefas \cite{stavropoulos2011survey}.

Oferecer um serviço a um utilizador é uma expressão muito vaga, dado que existe um enorme número de serviços que podem ser fornecidos. Assim deve ser tido em conta vários fatores relacionados com o utilizador e as suas características que definam o que realmente é útil e do interesse do utilizador e que é possível implementar dentro desse contexto \cite{stavropoulos2011survey}. Assim, a definição do Ambiente Inteligente deve ter em conta o âmbito dos seus utilizadores e ainda utilizar as preferências e informação relevante de cada um para que os serviços fornecidos estejam adaptados às suas necessidades e características. Um exemplo prático que demonstra a importância deste fator é o controlo do ar condicionado, ou seja, este deve ter em conta por exemplo o tipo de utilizadores porque idosos e crianças são normalmente mais sensíveis ao frio e mudanças de temperatura do que uma pessoa de meia idade.

Para além da importância da personalização que os serviços devem oferecer perante o utilizador que usufrui do serviço, deve ser tido ainda em conta a interação que este deve ter com o utilizador. Assim sendo, uma das características que um Ambiente Inteligente deve oferecer nos seus serviços é uma interação simples e intuitiva \cite{stavropoulos2011survey}. Esta premissa é fundamental para que os vários tipos de utilizadores consigam usufruir ao máximo do serviço que lhes é proposto e assim contribuir para o seu bem estar e comodidade.

\section{Monitorização}
A monitorização \cite{levin1999fundamentals, salber1999designing} corresponde a uma parte fundamental da ação de um Ambiente Inteligente. É esta funcionalidade de um Ambiente Inteligente, suportada pelo recurso a dispositivos tecnológicos próprios, que permite que o comportamento de um utilizador seja acompanhado e consequentemente seja possível retirar informação desse comportamento ao longo do período de tempo em que este é monitorizado.

A monitorização permite obter um conjunto de informação muito específico e de grande utilidade para um Ambiente Inteligente. A informação proveniente da monitorização de um utilizador apresenta o comportamento, interação e reações nesse período de tempo. Isto permite não só registar a evolução dos dados recolhidos mas também verificar as diferenças ocorridas em diferentes períodos de tempo e as alterações derivadas de diferentes contextos, sejam eles psicológicos, físicos ou até ambientais. Para além disso é ainda possível verificar a evolução do utilizador ao longo do tempo. A monitorização não é exclusiva do ser humano e do seu comportamento, sendo possível aplicar esta técnica por exemplo na monitorização da temperatura ambiente de um local, do número de automóveis num parque de estacionamento, ou até na deteção de incêndios.

A utilização de técnicas de monitorização sobre determinado contexto está umbilicalmente ligada ao recurso a dispositivos tecnológicos que permitam transformar e recolher a informação produzida. Estes dispositivos tecnológicos utilizados são os sensores e podem ser de diferentes tipos, permitindo assim a recolha de diversos tipos de informação. Esta abrangência de dados permite que a informação recolhida em determinado ambiente apresente elevado grau de precisão e consequentemente garanta maior fiabilidade ao sistema e às conclusões que deste se retirem.


\section{Sistemas de Monitorização}

Existe tanto a nível académico como da indústria um grande interesse e evolução no que diz respeito ao desenvolvimento de sistemas de monitorização. Assim, nesta secção são apresentados alguns exemplos deste tipo de sistemas que integram nas suas plataformas a utilização de sensores. 

\begin{description}
	\item[Meggitt Sensing System] A Meggit\footnote{http://www.meggittsensingsystems.com/} é uma empresa que se especializou no desenvolvimento de sistemas de monitorização. A sua grande área de ação é relativa à medição  de parâmetros físicos em ambientes externos de aeronaves, naves espaciais, geradores de energia, estações nucleares, de gás e petróleo e ainda testes laboratoriais. Nestas áreas de ação, foram desenvolvidos sistemas de monitorização com vários focos, como por exemplo, monitorização de motores, fluídos, trem de aterragem, vibração, combustão, aceleração, ritmo cardíaco, entre outros.

	\item[FiberWatch] O FiberWatch\footnote{http://www.halliburton.com/en-US/ps/stimulation/fiber-optic-monitoring/fiberwatch-fiber-optic-distributed-temperature-sensing-service.page} consiste num sistema composto por pontos de fibra óptica e sensores distribuídos que oferece um conjunto de serviços como o levantamento de distribuição de temperaturas e acústicos digitais em poços e ainda medidores de pressão e temperatura. Este sistema desenvolvido pela Halliburton apresenta características muito específicas tendo em conta as condições em que opera, isto é, estes serviços estão preparados para funcionar em ambientes onde se verifiquem altas temperaturas e em situações de estimulação de ácido de alta pressão ou de gravidade assistida por vapor de drenagem.

	\item[MSR Sense] O MSR Networked Embedded Sensing Toolkit\footnote{http://research.microsoft.com/en-us/projects/msrsense/} disponibiliza um conjunto de aplicações que permitem a recolha, processamento, armazenamento e consulta de dados provenientes da rede de sensores implementada. O software totalmente implementado na linguagem de programação C\#, contém uma biblioteca de implementação de algoritmos de processamento de sinais e deteção de eventos. Este sistema foi desenvolvido no âmbito da Microsoft Research, que consiste numa divisão da Microsoft com o intuito de desenvolver ideias e projetos que posteriormente venham a ser integrados nos seus produtos.

	\item[MISST]  O Multi-sensor Improved Sea Surface Temperature\footnote{http://www.misst.org/} consiste num sistema que através de satélites equipados com dispositivos que medem a radiação infravermelha e de microondas e ainda com recurso a boias à deriva no oceano e navios, conseguem recolher informação sobre a temperatura nos oceanos. Estas informações permitem tirar conclusões sobre mudanças e previsões climatéricas, efetuar estudos sobre a região do oceano em contacto com a atmosfera e a interação entre o ar e o mar. Este sistema foi desenvolvido pela Remote Sensing System especialista na monitorização remota da Terra através da utilização de sensores de micro-ondas colocados em satélites.

	\item[SiMoVe] O SiMoVe\footnote{http://www.digiwest.pt/simove/index.html} consiste num sistema que monitoriza veículos automóveis tendo em conta três vertentes: a sua localização geográfica, o estado do veículo e a sua dinâmica. Este sistema é assente na instalação de um dispositivo móvel composto por sensores no veículo que é responsável pela recolha e envio da informação para um computador central que trata do seu processamento e oferece uma interface para que o utilizador consiga visualizar os seus dados. Este sistema de monitorização pode também ser utilizado para controlo de frotas de veículos oferecendo aos seus utilizadores a informação necessária para que possam otimizar os seus percursos, efetuar prevenção sobre o funcionamento dos seus veículos e monitorizar o perfil de condução dos motoristas, permitindo assim rentabilizar o funcionamento da sua frota.


\end{description}


