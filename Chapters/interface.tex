\chapter{Interface Web}

Com arquitetura desenvolvida e funcional os utilizadores devem ter uma plataforma através da qual consigam acompanhar o seus registos resultantes da utilização da própria arquitetura. Essa foi uma das preocupações dessa dissertação, ou seja, fornecer também uma interface através da qual cada utilizador consiga aceder aos seus dados e várias informações relacionados com a monitorização efetuada com recurso à arquitetura. Serão apresentados assim neste capítulo a metodologia de desenvolvimento utilizada na concepção da interface web, os diferentes níveis de acesso existentes na plataforma, bem como as permissões a cada cada tipo de utilizador terá, os diversos registos de monitorização de cada utilizador e as diferentes formas de apresentação da informação, e ainda acesso ao histórico e evolução de cada utilizador.

\section{Metodologia/Padrão de Desenvolvimento}

No desenvolvimento da plataforma Web, através da qual os utilizadores poderão aceder aos seus registos de monitorização, foi utilizado como padrão de arquitetura de software o modelo MVC. Este padrão, cuja sigla MVC signfica \textit{Model-View-Controller}, trata-se de um dos padrões de desenvolvimento de software mais utilizados atualmente, e tem por base promover a reutilização do código desenvolvido, bem como a separação de conceitos e definição de interação entre eles em termos de arquitetura de software.

Este padrão torna-se extremamente útil, e consequentemente bastante utilizado, com o aumento da complexidade das aplicações a desenvolver. Este facto levou a que existisse a necessidade de separar conceitos de forma a simplificar o processo de desenvolvimento para os responsáveis pela construção destas plataformas. Desta forma, torna-se fundamental a separação entre os dados da aplicação e a camada de apresentação, garantindo que as alterações feitas no \textit{layout} não interferem com a manipulação dos dados e estes podem ser modificados sem que isso tenha repercussões no \textit{layout}. Esta separação apenas é possível pela existência de um componente responsável pela intermediação entre os dois, o controlador.

O modelo MVC, utilizado no desenvolvimento desta interface Web, divide assim os componentes da aplicação em três camadas: o \textit{Model} que representa os dados e o métodos de acesso e edição destes, a \textit{View} que corresponde à camada de visualização da aplicação, sendo a parte responsável pela iteração com os utilizadores. E por fim o \textit{Controller} que contém os métodos que processam os pedidos provenientes da \textit{view} e pode também invocar alterações no \textit{model}, funcionando como um controlador da aplicação e a ponte entre as duas camadas restantes. Desta forma é possível organizar de forma mais modular a arquitetura de uma aplicação web, centrando em cada camada única e exclusivamente as funções para as quais essa foi concebida, simplificando estruturalmente o desenvolvimento da aplicação.

\section{Tipos de Acesso}

A interface web criada para complementar a utilização da arquitetura foi desenhada com dois níveis de acesso para os seus utilizadores. Desta forma, os dois tipos de perfis existentes terão permissão de acesso a diferentes funcionalidades e informações da plataforma. Esta divisão de permissões em apenas dois níveis baseia-se na necessidade de ter um acesso para o utilizador comum consultar as suas informações e registos, e o outro acesso para o administrador conseguir consultar todos os conteúdo da disponíveis na plataforma.

O acesso com maior nível corresponde assim às permissões de administração. Este nível de acesso disponibiliza um conjunto alargado de informações sobre os utilizadores da arquitetura, dos seus registos de monitorização e ainda sobre os sensores conectados à arquitetura bem como o seu funcionamento. Este perfil de utilizador corresponde ao tipo de acesso mais elevado definido para esta interface e apenas um conjunto restrito de utilizadores devem ter acesso a estes conteúdos.

 \begin{figure}[htb]
   \centering
   \includegraphics[scale=0.29]{Images/panel.png}
   \caption{Página principal de Administrador}
\end{figure}

O outro nível de acesso implementado corresponde ao acesso atribuído aos utilizadores comuns da arquitetura. Através do seu acesso à interface os utilizadores da arquitetura conseguem aceder às suas informações pessoais e a todos os dados relacionados com a sua monitorização. Deste modo, tem acesso a todos os seus registos de monitorização apresentada de diferentes formas e pode ainda efetuar consultas sobre todo o seus histórico. Este nível representa um acesso mais restrito que o anterior, visto que limita os dados apresentados apenas aos correspondentes ao próprio utilizador e aos seus registos de monitorização.

 \begin{figure}[htb]
   \centering
   \includegraphics[scale=0.29]{Images/home.png}
   \caption{Página principal de Utilizador}
\end{figure}

\section{Histórico de User/Resultados Monitorização}

No contexto de utilização da arquitetura por parte de um utilizador, este sentirá grande interesse em poder consultar e analisar o que resultou da sua monitorização. Deste modo, uma das preocupações no desenvolvimento da interface foi fornecer aos utilizadores acesso a todo o seu histórico de registos provenientes da utilização da arquitetura desenvolvida. Cada utilizador terá assim na sua área pessoal acesso total a todo o conteúdo de informação recolhida e processada pela arquitetura. Para além disso terá ainda disponível diversas formas de acesso e apresentação da informação, com o intuito de que os seus utilizadores possam tirar o máximo de proveito do seu histórico de registos.

Assim para além das informações mais genéricas apresentados a cada utilizador no seu painel principal, este poderá aceder a duas áreas da interface nas quais poderá consultar todos os seus registos de monitorização. A diferenças entre estas duas áreas reside na forma de apresentação da informação. Na primeira área o utilizador poderá aceder aos seus registos sob a forma de tabela, selecionando o tipo de dados que pretende consultar e uma métrica associada ao tipo de dado escolhido. No caso de esse tipo de dados não conter nenhuma métrica disponível na arquitetura a informação será apresentada em bruto, tal como foi recolhida pelo dispositivo responsável pela monitorização. Para além disso, o utilizador poderá ainda escolher o período de monitorização que pretende, existente relativamente aos dois parâmetros já definidos.

 \begin{figure}[htb]
   \centering
   \includegraphics[scale=0.29]{Images/tables.png}
   \caption{Página de tabelas de registos de monitorização da interface}
\end{figure}

A segunda área de apresentação dos registos de monitorização disponibilizada trata-se da apresentação de resultados através de gráficos. Este forma de apresentação é bastante útil para os utilizadores pois permitem-lhe ter uma noção gráfica mais clara dos seus registos e variações comportamentais ao longo do período de monitorização. Esta área oferece assim uma forma mais simples de visualização e interpretação dos resultados. Tal como na área composta por tabelas, o utilizador poderá selecionar o tipo de dados, uma métrica associada e a data do período a visualizar. Neste caso em particular apenas os tipos de dados com métricas associadas serão apresentados devido à dificuldade em gerar gráficos através de informação em bruto.

 \begin{figure}[htb]
   \centering
   \includegraphics[scale=0.29]{Images/graphs.png}
   \caption{Página de gráficos de registos de monitorização da interface}
\end{figure}


\section{Administração/Funcionalidades}

Relativamente às funcionalidades de administração, estas residem sobretudo no acesso a informação privilegiada. Tem como principal objetivo acompanhar a utilização da arquitetura e poder aceder a informações relativas a essa utilização. Desta forma, para além dos dados estatísticos apresentados no painel de administração, no qual pode, por exemplo, saber que utilizadores estão a ser monitorizados e quais os sensores utilizados no momento, o administrador tem acesso a outros conteúdos relevantes.

\begin{figure}[htb]
   \centering
   \includegraphics[scale=0.29]{Images/users.png}
   \caption{Página de consulta de utilizadores da arquitetura}
\end{figure}

Cada administrador do sistema poderá consultar os dados de todos os utilizadores do sistema, podendo ver não só os seus dados, como também saber o número de monitorizações efetuado por cada um. Para além das informações relativas a utilizadores da arquitetura, o administrador tem também acesso as registos dos sensores utilizados na arquitetura, independentemente do momento da sua utilização e ainda à lista de todas as métricas integradas na arquitetura. Todo este conjunto de funcionalidades permite a cada administrador acompanhar a utilização da arquitetura e aceder a dados fundamentais sobre esta.

\begin{figure}[htb]
   \centering
   \includegraphics[scale=0.29]{Images/sensores.png}
   \caption{Página de consulta de sensores registados na arquitetura}
\end{figure}


\section{Conclusão}

O padrão de desenvolvimento utilizado no desenvolvimento desta plataforma revelou-se bastante útil e contribui para o rápido desenvolvimento da interface. A separação de componentes da aplicação e tratamento de cada função em específica na respetiva camada contribui para a simplificação do desenvolvimento da interface e consequentemente tornou o processo mais rápido e intuitivo. Apesar de não se tratar de uma aplicação com elevada complexidade, o potencial e capacidade deste padrão ficou bem patente no processo de desenvolvimento e demonstrou o porquê da utilização massiva deste padrão atualmente.

Quanto à interface construída, esta terá grande utilidade não só para os seus utilizadores e da arquitetura, como também para os administradores da aplicação. O objetivo consistia em abranger ao máximo a arquitetura e a sua informação, não só resultante da monitorização mas também toda a que diretamente está relacionada com isso e com o funcionamento da arquitetura.  Objetivo esse que de certa forma foi alcançado, sendo possível aos seus utilizadores aceder aos dados mais relevantes da arquitetura consoante o seu nível de acesso à interface.

No caso concreto da apresentação dos dados relativos à monitorização e estatísticas diretamente relacionadas, estas foram implementadas da forma idealizada. Os dados estatísticos presentes nas páginas principais de cada tipo de utilizador dão pequenas informações importantes sobre a utilização da arquitetura, tanto relativamente ao utilizador em questão como ao funcionamento de todos os utilizadores. Quanto às formas de apresentação dos resultados de monitorização foram desenvolvidas com o intuito de permitir ao utilizador aceder de forma mais genérica e de fácil visualização e interpretação aos registos através dos gráficos. E no caso do utilizador pretender uma verificação e análise mais pormenorizada através das tabelas de registos. Todas estas funcionalidades foram implementadas com o intuito de permitir um acesso simples e prático a este conjunto de dados fundamentais.

Assim pode-se concluir da interface criada que se trata de uma ferramenta extremamente útil e que serve como complemento à arquitetura e às suas funcionalidades. Tendo por base uma arquitetura que permite a recolha de diferentes tipos de informação de forma distribuída e tratamento e agregação da mesma, havia a necessidade de criar uma ferramenta de consulta dos dados resultantes. Através da interface desenvolvida os utilizadores da arquitetura podem assim consultar de forma simples e intuitiva o resultado das suas monitorizações e informação relacionada. Esta interface web permite também comprovar a funcionalidade da arquitetura e verificar a sua capacidade de recolha e gestão da informação.

