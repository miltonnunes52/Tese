% !TEX encoding = IsoLatin
%\thispagestyle{empty}
\begin{center}
{\large \textit{Resumo} }
\end{center}
%\newenvironment{resumo}
%{\thispagestyle{empty} \vspace*{\stretch{1}}\begin{flushleft} \em}{\end{flushleft} \vspace*{\stretch{3}}}


\vspace{0.5cm}
\quad Existe atualmente uma grande facilidade no acesso � informa��o e um interesse generalizado em utilizar essa informa��o para aumentar a comodidade e bem estar das pessoas. O conhecimento � reconhecidamente uma arma poderosa que tende ser explorada at� ao limite. Desta forma importa encontrar todas as fontes de informa��o que possam ser �teis para alcan�ar esse objetivo global. Neste sentido o potencial da informa��o ambiente que rodeia um ser humano � enorme e a falta de aproveitamento verificada sobre esse conhecimento � uma falha que deve ser colmatada, sendo poss�vel aplic�-la de modo a proporcionar melhores condi��es de vida e de bem estar. Pretende-se assim com este trabalho desenvolver uma arquitetura composta sensores colocados de forma distribu�da capaz de monitorizar ambientes inteligentes com base em diversos fatores. Nesta arquitetura a fun��o de cada sensor pode ser visto como um servi�o prestado aos seus utilizadores, atrav�s dos quais s�o recolhidos determinado tipo de informa��o que posteriormente ser�o disponibilizados a cada utilizador. Desta forma, este trabalho fornece ao ser humano uma vis�o global, baseada em v�rios indicadores recolhidos, sobre a sua intera��o e rea��o ao longo do per�odo de monitoriza��o.
\vspace{3cm}

\textbf{Palavras-chave:} Arquitectura, Servi�os, Ambientes Inteligentes, Monitoriza��o, Sensores

%\end{resumo}

